\documentclass[unicode]{beamer}
%\usetheme{CambridgeUS}
%\usetheme{Boadilla}
%\usetheme{Berkeley}
\usetheme{Goettingen}
%\usecolortheme{beaver}
\usecolortheme{seahorse}     %цветовая схема
%\useoutertheme{smoothbars}
\useinnertheme{circles}   %внутренняя тема
\usefonttheme{serif}    %шрифты

\usepackage[utf8]{inputenc}
\usepackage[T1]{fontenc}
\usepackage[russian]{babel}


\title[Курсовая работа]{Построение решений типа бегущей волны уравнений Бюргерса и Кортвега-де Фриза}
\author[М.А.~Михайлов]{М.А.~Михайлов}
\institute[]{МГТУ им. Н.Э. Баумана}
\date{\today}

\begin{document}

\begin{frame}
\titlepage
\end{frame}

\begin{frame}
\frametitle{Содержание}
\tableofcontents
\end{frame}

\section[Постановка задачи]{Постановка задачи}
Уравнение Бюргерса имеет вид:
\begin{equation} \label{Bur}
u_t+uu_x={\mu}u_{xx},
\end{equation}
а Кортвега-де Фриза:
\begin{equation} \label{Court}
u_t+uu_x+{\varepsilon}u_{xxx}=0.
\end{equation}
Будем искать решения этих уравнений типа бегущей волны:
\begin{equation} \label{WF}
u(t,x)=v(y),\quad y=x-at.
\end{equation}
В первом случае поставим краевые условия вида:
\begin{equation}\label{starteqB}
v(y) = 
 \begin{cases}
   1,\quad y\to-{\infty}, \\
   0,\quad y\to+{\infty},
 \end{cases}
\end{equation}
и во втором -- вида:
\begin{equation} \label{starteqC}
v(y) = 0,\quad y\to{\infty}.
\end{equation}


\section{ Решение уравнения Бюргерса}
Для уравнения Бюргерса \eqref{Bur} получаем дифференциальное  уравнение второго порядка:
\begin{equation} \label{DUB}
-av^\prime+vv^\prime={\mu}v^{\prime\prime}.
\end{equation}
Cделав замену $v^\prime=p(y)$, $v^{\prime\prime}=p^\prime p$, получим дифференциальное уравнение первого порядка:
\begin{equation} \label{DUB1}
-ap+pv={\mu}pp^\prime.
\end{equation}
\[
\mu p=-av+\frac{v^2}{2},
\]
\begin{equation} \label{DUBI}
\mu v^\prime=-av+\frac{v^2}{2}.
\end{equation}
Подставив \eqref{starteqB}  в \eqref{DUBI}, найдем $a$:
\[
-a+\frac{1}{2}=\mu\cdot0, \quad\Rightarrow\quad a=\frac{1}{2}
\]
Следовательно, \eqref{DUBI} принимает вид:
\begin{equation} \label{DUBI1}
\frac{-v}{2}+\frac{v^2}{2}={\mu}\frac{dv}{dy}
\end{equation}
Решением \eqref{DUB1} является:
\begin{equation} \label{SolveD}
v=\frac 1{1+e^{\frac{ y}{2\mu}}}.
\end{equation}
При помощи системы компьютерной алгебры Wolfram Mathematica  построим график решения \eqref{SolveD}.
\begin{figure}[H]
\centering
\includegraphics[scale=0.8]{B2D.pdf}
\caption{Решение уравнения Бюргерса.}
\label{ris:image}
\end{figure}
Подставим \eqref{SolveD} в формулу $u(t,x)=v(x-at)$. Получим решение:  $u(t,x)=\frac1{1+e^{\frac{1}{2\mu}\left(\frac{-t}{2}+x\right)}}.$

\begin{figure}[H]
\centering
\includegraphics[scale=0.4]{bu3D.png}
\caption{Решение уравнения Бюргерса типа бегущей волны.}
\label{ris:imageB}
\end{figure}
\newpage
Представим графики решения при фиксированных $t$ и $\mu$.
\begin{figure}[H]
\centering
\includegraphics[scale=0.5]{B0.pdf}
\caption{График решения при $t=0$.}
\label{ris:imageBt0}
\end{figure}
\begin{figure}[H]
\centering
\includegraphics[scale=0.5]{B50.pdf}
\caption{График решения при $t=50$.}
\label{ris:imageBt50}
\end{figure}
\newpage
\section{ Решение уравнения Кортвега-де Фриза}
Для уравнения Кортвега-де Фриза \eqref{Court} получим нелинейное дифференциальное уравнение третьего порядка:
\begin{equation} \label{DUC}
-av^\prime+vv^\prime+{\varepsilon}v^{\prime\prime\prime}=0.
\end{equation}
Уравнение один раз легко интегрируется:
\begin{equation} \label{DUC1}
-av+\frac {v^2}{2}+{\varepsilon}v^{\prime\prime}+c_1=0,
\end{equation}
Сделав замену $v^\prime(y)=p(y)$, $v^{\prime\prime}=p^\prime p$, получим дифференциальное уравнение первого порядка:
\begin{equation} \label{DUC2}
pp^\prime=\frac{av}{\varepsilon}-\frac{v^2}{2\varepsilon},
\end{equation}
\[
\frac {p^2}2= \frac{av^2}{2\varepsilon}-\frac{v^3}{6\varepsilon},
\]
\begin{equation} \label{DUCI}
 v^\prime=\sqrt{\frac{av^2}{\varepsilon}-\frac{v^3}{3\varepsilon}}.
\end{equation}
Решением \eqref{DUCI} будет являться:
\begin{equation} \label{SolveC}
v=\frac{3a}{\ch^2\left(\frac {\sqrt{\frac{a}{\varepsilon}}y}{2}\right)}.
\end{equation}
Построим график решения \eqref{SolveC}.
\begin{figure}[H]
\centering
\includegraphics[scale=0.8]{COURT.pdf}
\caption{Решение уравнения Кортвега-де Фриза при различных $a$ и $\varepsilon$.}
\label{ris:imageC2D}
\end{figure}
Подставив  $y=x-at$ в соотношение \eqref{SolveC}, получим решение задачи \eqref{Court}, \eqref{starteqC} типа бегущей волны:
\begin{equation} \label{RUN}
u(t,x)=\frac{3a}{\ch^2\left(\frac{\sqrt{\frac{a}{\varepsilon}}(x-at)}{2}\right)}.
\end{equation}
\begin{figure}[H]
\centering
\includegraphics[scale=0.25]{Court3D.png}
\caption{Решение уравнения Кортвега-де Фриза типа бегущей волны.}
\label{ris:imageC3D}
\end{figure}
Представим графики решения \eqref{RUN} при фиксированных $t$ и $\varepsilon$.
\begin{figure}[H]
\centering
\includegraphics[scale=0.9]{C1.pdf}
\caption{График решения при $t=1$.}
\label{ris:imageCt1}
\end{figure}
\begin{figure}[H]
\centering
\includegraphics[scale=0.9]{C50.pdf}
\caption{График решения при $t=50$.}
\label{ris:imageCt50}
\end{figure}
\section{Заключение}
\newpage
Аналитически найдены решения типа бегущей волны уравнений Бюргерса и Кортвега-де Фриза. Исследована зависимость полученных решений от параметров $\mu$ и $\varepsilon$ соответственно. Построены графики решений в системе компьютерной алгебры Wolfram Mathematica.

% Литература %
\newpage
\section{Список использованных источников}
\begin{thebibliography}{12}
\bibitem{Ryskin} Рыскин Н.\,М., Трубецков Д.\,И. Нелинейные волны / М.: <<Наука>>, 1984. 306 с.
\bibitem{Shapiro} Шапиро Д.\,А. Уравнения в частных производных. Специальные функции. Асимптотики  / М.: Изд-во НГУ, 2004. 123 с
\bibitem{Elsholz} Эльсгольц Л.\,Э.  Дифференциальные уравнения и вариационное исчисление. / М.: <<Наука>>, 1965. 424 с.
\bibitem{Filippov} Филиппов А.\,Ф. Сборник задач по дифференциальным уравнениям. / М.: ЛЕНАНД, 2019. 240 с.
\end{thebibliography}
\newpage







\vspace{1mm}\vspace{1mm}\vspace{1mm}\vspace{1mm}\vspace{1mm}\vspace{1mm}\vspace{1mm}\vspace{1mm}\vspace{1mm}\vspace{1mm}\vspace{1mm}\vspace{1mm}\vspace{1mm}\vspace{1mm}\vspace{1mm}\vspace{1mm}\vspace{1mm}\vspace{1mm}\vspace{1mm}\vspace{1mm}\vspace{1mm}\centering \bf{Спасибо за внимание!}
\end{document}