\documentclass[12pt,a4paper]{article} 

\usepackage{../../fn2kursstyle}
\usepackage[russian]{babel}
\usepackage[T2A]{fontenc} 
\usepackage[utf8]{inputenc} 
\usepackage{geometry}
\usepackage{graphicx}
\usepackage{mathtools}
\usepackage{tikz}
\usepackage{pdfpages}
\usepackage{booktabs}
\usepackage{multirow,array}
\usepackage{siunitx}
\usepackage{amsmath}
\usepackage[hidelinks]{hyperref}

\counterwithout{equation}{section}
\counterwithout{figure}{section}
\graphicspath{{pic/}}
\frenchspacing 

\newcolumntype{C}[1]{>{\centering\arraybackslash}p{#1}}

\newcommand{\picref}[1]{рис. \ref{#1}}
\newcommand{\tabref}[1]{таблица \ref{#1}}
\newcommand{\half}{\frac{1}{2}}
\newcommand{\dhalf}{\dfrac{1}{2}}
\newcommand*{\Scale}[2][4]{\scalebox{#1}{$#2$}}

\title{Лабораторная работа №1 по дисциплине "Разработка программных комплексов" на тему "Методы конечных элементов"}
\group{ФН2-72Б}
\author{Токарев А.\,И.}
\supervisor{Азметов Х.\,Х.}
\date{2022}

\begin{document}
    \maketitle
    \tableofcontents
    \pagebreak

    \section{Задача}

    Создать программу решения дифференциального уравнения проекционными методами. Задано урванение на области $[0, 1]\colon$
    \[
        \dfrac{d^2 u}{dx^2} + u + x = 0, \quad u(0) = u(1) = 0.
    \]  

    Необходимо реализовать методы решения:
    \begin{enumerate}
        \item Метод коллокаций в точках
        \item Метод коллокаций в подобластях
        \item Метод Бубнова-Галеркина
        \item Метод Галеркина
        \item Метод наименьших квадратов
        \item Метод Ритца
    \end{enumerate}

    Для каждого из методов нужно получить решение с порядком аппроксимации от $1$ до $3$.

    \pagebreak

    \section{Метод коллокации в точке}

    \pagebreak

    \section{Метод коллокаций в подобластях}

    \pagebreak

    \section{ Метод Бубнова-Галеркина}

    \pagebreak

    \section{ Метод Галеркина}

    \pagebreak

    \section{Метод наименьших квадратов}

    \pagebreak

    \section{Метод Ритца}

\end{document}